\documentclass[10pt]{beamer}

\usetheme{metropolis}
\usepackage{appendixnumberbeamer}

% German
\usepackage[utf8]{inputenc}
\usepackage[T1]{fontenc}
\usepackage[ngerman]{babel}

\usepackage{booktabs}
\usepackage[scale=2]{ccicons}

\usepackage{pgfplots}
\usepgfplotslibrary{dateplot}

\usepackage{xspace}
\newcommand{\themename}{\textbf{\textsc{metropolis}}\xspace}

% graphics path
\graphicspath{{images/}}

% set KIT color scheme
\definecolor{kitGreen}{HTML}{32A189}
\setbeamercolor{alerted text}{fg=kitGreen}

% use fdsymbols and math font
\usepackage{FiraSans}
% \usefonttheme[onlymath]{serif}
% \usepackage{eulervm}

% coloneqq
\usepackage{mathtools}

% Math helpers
\newcommand{\R}{\mathbb{R}}
\newcommand{\N}{\mathbb{N}}
\renewcommand{\min}{\text{min}}
\renewcommand{\max}{\text{max}}

% wide tilde by default
\renewcommand{\tilde}[1]{\widetilde{#1}}

% my favourite greek symbols
\let\origphi\phi
\let\phi\varphi
\let\origtheta\theta
\let\origepsilon\epsilon
\let\epsilon\varepsilon
\let\origupepsilon\upepsilon
\let\upepsilon\upvarepsilon

% better multiplication symbol
\mathcode`\*="8000
{\catcode`\*\active\gdef*{\cdot}}

% use commas instead of points in math environments ("german notation")
\DeclareMathSymbol{.}{\mathord}{letters}{"3B}

% Footer
\setbeamertemplate{frame footer}{\textcolor{gray}{\tiny{Jens Ochsenmeier, Seminar Fraktale Geometrie}}}

% fill blocks
\metroset{block=fill}

% smaller captions
\usepackage[font={color=gray,scriptsize,it}]{caption}

\title{Seminar Fraktale}
\subtitle{Fraktale Bildcodierung}
\date{\today}
\author{Jens Ochsenmeier}
\titlegraphic{\hfill\includegraphics[height=1cm]{logo.png}}

\begin{document}

\maketitle

% \begin{frame}{Table of contents}
%   \setbeamertemplate{section in toc}[sections numbered]
%   \tableofcontents[hideallsubsections]
% \end{frame}

\section{Wichtige Werkzeuge}

%
% SLIDE 1: Recap metrische Räume
%
\begin{frame}{Metrische Räume}
  Menge \( X \) mit reellwertiger Funktion \( d: X \times X \to \R \), die
  \begin{enumerate}
    \item \alert{symmetrisch} und
    \item \alert{positiv definit} ist und die
    \item \alert{Dreiecksungleichung} erfüllt 
  \end{enumerate}
  \pause{}
  \begin{itemize}
    \item[\( \to \)] \emph{vollständige} metrische Räume (Cauchy-Folgen konvergieren)
    % NOTE
    % Gegenbeispiel vollständiger Raum: Q mit Betragsmetrik d(x,y) = |x-y|
    \pause{}
    \item[\( \to \)] \emph{kompakte} metrische Räume (beschränkt und abgeschlossen)
    % NOTE
    % Gegenbeispiel kompakter Raum: R^{n x n} mit Matrixnorm ist vollständig, aber nicht kompakt (weil nicht abgeschlossen)
  \end{itemize}
\end{frame}

%
% SLIDE 2: Recap Raum der Fraktale
%
\begin{frame}{Raum der Fraktale}
  Raum der Fraktale über \( X \):
  \begin{equation*}
    \mathcal{H}(X) \coloneqq \left \{ \mathcal{A} \subseteq X : \mathcal{A} \text{ kompakt} \right \} \setminus \varnothing
  \end{equation*}
  \pause{}
  \alert{Abstandsbegriffe}:
  \begin{itemize}
    \item \( d(x,A) \coloneqq \min\left \{ d(x,a) : a \in A \right \} \) \quad \ (\( x \in X \), \( A \in \mathcal{H}(X) \))
    \item \( d(A,B) \coloneqq \max\left \{ d(x,B) : x \in A \right \} \) \quad (\( A, B \in \mathcal{H}(X) \))
    \pause{}
    \item \( h(A,B) \coloneqq \max\left \{ d(A,B),\ d(B,A) \right \} \) (Hausdorff-Abstand)
  \end{itemize}
  \pause{}
  \( \to \) \alert{Raum der Fraktale} \( (\mathcal{H}(X), h) \)
\end{frame}

\begin{frame}{Fraktal}
  \begin{quote}
    Ein Fraktal ist eine Teilmenge von \( (\mathcal{H}(X), h) \).
  \end{quote}
\end{frame}

\section{Abbildungen und Transformationen}

\begin{frame}{Kontraktion}
  Abbildung \( \phi: X \to X \) für die \( c \in [0,1) \) existiert, sodass
  \begin{equation*}
    \forall \ x,y \in X : d(\phi(x), \phi(y)) \leq c * d(x,y)
  \end{equation*}
  \alert{Kontraktionsfaktor}: \( c \)
\end{frame}

\begin{frame}{Ähnlichkeitsabbildung}
  Abbildung \( \phi: X \to X \) für die \( c \in [0,1) \) existiert, sodass
  \begin{equation*}
    \forall \ x,y \in X : d(\phi(x), \phi(y)) = c * d(x,y)
  \end{equation*}
  \alert{Kontraktionsverhältnis}: \( c \)
\end{frame}

\begin{frame}{Banachscher Fixpunktsatz}
  \begin{columns}[T,onlytextwidth]
    \column{0.6\textwidth}
      \begin{minipage}[c][.4\textheight][c]{\linewidth}
        \begin{itemize}
          \item Kontraktion \( \phi : X \to X \)
          \item Iterative Folge \( {\left( x_n \right)}_{n \in \N} \), \ \( x_{n+1} = \phi\left( x_n \right) \)
          \item \emph{Beliebiger} Startwert \( x_0 \)
        \end{itemize}
      \end{minipage}
    
    % NOTE
    % Linke obere Ecke "verfolgen" --> geht in Richtund Mitte
    \column{0.4\textwidth}
       \only<2>{
         \begin{figure}
          \newcounter{density}
          \setcounter{density}{20}
          \begin{tikzpicture}[scale=.6]
            \def\couleur{alerted text.fg}
            \path[coordinate] (0,0)  coordinate(A)
                        ++( 90:5cm) coordinate(B)
                        ++(0:5cm) coordinate(C)
                        ++(-90:5cm) coordinate(D);
            \draw[fill=\couleur!\thedensity] (A) -- (B) -- (C) --(D) -- cycle;
            \foreach \x in {1,...,10}{%
                \pgfmathsetcounter{density}{\thedensity+20}
                \setcounter{density}{\thedensity}
                \path[coordinate] coordinate(X) at (A){};
                \path[coordinate] (A) -- (B) coordinate[pos=.10](A)
                                    -- (C) coordinate[pos=.10](B)
                                    -- (D) coordinate[pos=.10](C)
                                    -- (X) coordinate[pos=.10](D);
                \draw[fill=\couleur!\thedensity] (A)--(B)--(C)-- (D) -- cycle;
            }
          \end{tikzpicture}
        \end{figure}
       }
  \end{columns}
  \begin{block}{Satz (Banach, 1922)}
    \begin{enumerate}
      \item Es existiert genau ein \( \tilde{x} \in X \), sodass \( \phi(\tilde{x}) = \tilde{x} \).
      \item Es ist \( \underset{n \to \infty}{\lim} x_n = \tilde{x} \).
    \end{enumerate}
  \end{block}
\end{frame}

\begin{frame}{Lineare Transformation}
  \alert{Lineare Transformation}
  \begin{itemize}
    \item bildet Geraden auf Geraden ab
    \item fixiert den Ursprung
  \end{itemize}
  Darstellung durch Matrix:
  \begin{equation*}
    t: \R^2 \to \R^2, \qquad t(x,y) \coloneqq \begin{pmatrix}
      a & b \\
      c & d
    \end{pmatrix} * \begin{pmatrix}
      x \\ y
    \end{pmatrix}
  \end{equation*}
  Konkret: Skalieren, Spiegeln, Strecken, Drehen
\end{frame}

\begin{frame}{Affine Transformation}
  \alert{Affine Transformation}: Lineare Transformation + Translation
  
  Darstellung durch Matrix:
  \begin{equation*}
    w: \R^2 \to \R^2, \qquad w(x,y) \coloneqq A*x + t = \underbrace{\begin{pmatrix}
      a & b \\
      c & d
    \end{pmatrix} * \begin{pmatrix}
      x \\ y
    \end{pmatrix}}_{\text{lineare Transformation}} + \underbrace{\begin{pmatrix}
      e \\ f
    \end{pmatrix}}_{\text{Translation}}
  \end{equation*}
\end{frame}

\begin{frame}{Affine Transformation --- Konstruktion}
  \begin{enumerate}
    \item Drei Punkte auf Startmenge wählen
    \item Dazugehörige Punkte in Bildmenge finden
    \item Ein paar Gleichungssysteme lösen
    \begin{align*}
      \left( \begin{smallmatrix}
        a & b \\ c & d
      \end{smallmatrix} \right) \left( \begin{smallmatrix}
        x_1 \\ x_2
      \end{smallmatrix} \right) + \left( \begin{smallmatrix}
        e \\ f
      \end{smallmatrix} \right) &= \left( \begin{smallmatrix}
        \widetilde{x}_1 \\ \widetilde{x}_2
      \end{smallmatrix} \right) \\
      \left( \begin{smallmatrix}
        a & b \\ c & d
      \end{smallmatrix} \right) \left( \begin{smallmatrix}
        y_1 \\ y_2
      \end{smallmatrix} \right) + \left( \begin{smallmatrix}
        e \\ f
      \end{smallmatrix} \right) &= \left( \begin{smallmatrix}
        \widetilde{y}_1 \\ \widetilde{y}_2
      \end{smallmatrix} \right) \\
      \left( \begin{smallmatrix}
        a & b \\ c & d
      \end{smallmatrix} \right) \left( \begin{smallmatrix}
        z_1 \\ z_2
      \end{smallmatrix} \right) + \left( \begin{smallmatrix}
        e \\ f
      \end{smallmatrix} \right) &= \left( \begin{smallmatrix}
        \widetilde{z}_1 \\ \widetilde{z}_2
      \end{smallmatrix} \right)
    \end{align*}
    \item[\( \to \)] \alert{affine Transformation}
  \end{enumerate}
\end{frame}

\section{Iterierte Funktionensysteme}

\begin{frame}{Iteriertes Funktionensystem (IFS)}
  \begin{columns}[T,onlytextwidth]
    \column{.7\textwidth}
      \alert{IFS}: Familie \( \left \{ \Phi_1, \dots, \Phi_n \right \} \) mit \( \Phi_i : X \to X \) Kontraktion \\
      \scriptsize{\( (X,d) \) vollständiger metrischer Raum}
    
      \normalsize{}
      \alert{Kontraktionskoeffizient} \( c = \max\left \{ c_1, \dots, c_n \right \} \) \\
      \scriptsize{(\( c_i \coloneqq \) Kontraktionsfaktor von \( \Phi_i \))}
    
      \normalsize{}
      \alert{Attraktor} des IFS:\@ \( C \subset X \) derart, dass
      \begin{equation*}
        \bigcup_{i=0}^n \Phi_i(C) = C
      \end{equation*}

    \column{.3\textwidth}
      % NOTE
      % http://jsbin.com/temudexazo/edit?js,output
      \begin{figure}
        \includegraphics[width=.8\textwidth]{fern}
      \end{figure}
  \end{columns}
\end{frame}

\section{Modellierung echter Bilder}

\begin{frame}{``Echte'' Bilder}
  \alert{Echtes Bild}: \emph{Etwas, dass man irgendwo sehen oder sich zumindest zu sehen vorstellen könnte}

  \begin{columns}[T,onlytextwidth]
    \column{.5\textwidth}
      \begin{itemize}
        \item \( \mathcal{R} \): Menge aller echten Bilder
        \item \( \mathcal{I} \in \mathcal{R} \): echtes Bild
      \end{itemize}

      \begin{itemize}
        \pause{}
        \item[\( \to \)] mathematisch nicht wirklich greifbar \\
        \pause{}
        \item[\( \to \)] Modellierung entlang von Eigenschaften
      \end{itemize}
      \pause{}
    \column{.45\textwidth}
      \begin{figure}
        \includegraphics[width=\textwidth]{neuschwanstein}
        \caption*{\tiny{Schloss Neuschwanstein, Unsplash}}
      \end{figure}
  \end{columns}
\end{frame}

\begin{frame}{Eigenschaft 1}
  Ein Bild besitzt
  \begin{enumerate}
    \item \alert{Träger} \  \( \square = [a,b] \times [c,d] \subset \R^2 \)
    \item \alert{Maße} \  \( (b-a) \) und \( ( c-d ) \)
  \end{enumerate}

  \begin{itemize}
    \item[\( \to \)] Einbettung in den euklidischen Raum
  \end{itemize}
\end{frame}

\begin{frame}{Eigenschaft 2}
  Ein Bild besitzt \alert{chromatische Werte}
  \begin{equation*}
    c: \square \to \R
  \end{equation*}
  Hier: \textbf{nur Graustufen}: \( c \) ist die \emph{Helligkeit} an einem bestimmten Punkt

  \begin{minipage}{.45\textwidth}
    \begin{figure}
      \includegraphics[width=\textwidth]{neuschwanstein-sw}
    \end{figure}
  \end{minipage}
  \hfill
  \begin{minipage}{.45\textwidth}
    \begin{figure}
      \includegraphics[width=\textwidth]{neuschwanstein-chromatic}
    \end{figure}
  \end{minipage}
\end{frame}

\begin{frame}{Eigenschaft 3}
  \( \mathcal{I} \in \mathcal{R} \) besitzt \emph{keine} Auflösung!

  \( \to \) verschiedene Auflösungen beschreiben dasselbe Bild

  \begin{minipage}{.45\textwidth}
    \begin{figure}
      \includegraphics[width=\textwidth]{neuschwanstein}
    \end{figure}
  \end{minipage}
  \hfill
  \begin{minipage}{.45\textwidth}
    \begin{figure}
      \includegraphics[width=\textwidth]{neuschwanstein-pixelated}
    \end{figure}
  \end{minipage}
\end{frame}

\begin{frame}{Eigenschaft 4}
  \( \mathcal{R} \) ist unter Zuschneiden von Bildern abgeschlossen

  \begin{minipage}{.45\textwidth}
    \begin{figure}
      \includegraphics[width=\textwidth]{neuschwanstein}
    \end{figure}
  \end{minipage}
  \hfill
  \begin{minipage}{.45\textwidth}
    \begin{figure}
      \includegraphics[width=\textwidth]{neuschwanstein-cropped}
    \end{figure}
  \end{minipage}
\end{frame}

\begin{frame}{Eigenschaft 5}
  \( \mathcal{R} \) ist unter affinen Transformationen abgeschlossen

  \begin{minipage}{.375\textwidth}
    \begin{figure}
      \includegraphics[width=\textwidth]{neuschwanstein}
    \end{figure}
  \end{minipage}
  \hfill
  \begin{minipage}{.525\textwidth}
    \begin{figure}
      \includegraphics[width=\textwidth]{neuschwanstein-skewed}
    \end{figure}
  \end{minipage}
\end{frame}

\begin{frame}{Digitalisierung}
  \begin{equation*}
    \mathcal{I} \in \mathcal{R} \leadsto \text{Pixel}
  \end{equation*}

  \begin{minipage}{.45\textwidth}
    \begin{itemize}
      \item Welche Pixel kriegen welche Farbe?
      \item Interpolation?
      \item Skalierung?
      \item[\( \to \)] Maßtheorie,\dots
    \end{itemize}
  \end{minipage}
  \hfill
  \begin{minipage}{.45\textwidth}
    \begin{figure}
      \includegraphics[width=\textwidth]{neuschwanstein-rasterization}
    \end{figure}
  \end{minipage}

  \( \Rightarrow \) Gerastertes Bild mit Farben pro Pixel
\end{frame}

\section{Collage-Theorem}

\begin{frame}{Collage-Theorem}
  \begin{itemize}
    \item vollständiger metrischer Raum \( (X,d) \)
    \item \( L \in \mathcal{H}(X) \)
    \item \( \epsilon \geq 0 \)
    \item IFS \( \left \{ \Phi_1,\dots,\Phi_n \right \} \)
    \begin{itemize}
      \item Kontraktionskoeffizient \( 0 \leq s < 1 \) 
      \item Attraktor \( A \)
    \end{itemize}
  \end{itemize}

  \begin{block}{Satz (Barnsley, 1992)}
    \begin{align*}
      h\left( L,\bigcup_{i=0}^n \Phi_i(L) \right) \leq \epsilon \Rightarrow h(L,A) \leq \frac{\epsilon}{1 - s}
    \end{align*}
  \end{block}
\end{frame}

\section{Stetige Abhängigkeit von Parametern}

\begin{frame}{Stetige Abhängigkeit --- Fixpunkt}
  \begin{itemize}
    \item \( (P, d_P) \) metrischer Raum
    \item \( (X,d) \) vollständiger metrischer Raum
    \item \( \Phi: P \times X \to X \) Familie von Kontraktionen auf \( X \)
    \item \( 0 \leq s < 1 \) Kontraktionskoeffizient von \( \Phi(p, \cdot) \)
    \item \( \Phi(\cdot, x) \) stetig für jedes \( x \in X \)
  \end{itemize}
  \begin{block}{Lemma}
    Der Fixpunkt \( \tilde{x}(p) \) von \( \Phi \) ist stetig abhängig von \( p \). Das heißt:
    \begin{equation*}
      \tilde{x} : P \to X \ \text{ist stetig.}
    \end{equation*}
  \end{block}
\end{frame}

\begin{frame}{Stetige Abhängigkeit --- Attraktor}
  \begin{itemize}
    \item \( (X,d) \) und \( (P, d_P) \) wie gehabt
    \item IFS wie gehabt
    \item \( \Phi_i \) hänge von \( p \) im Sinne folgender Einschränkung ab:
    \begin{equation*}
      d(\Phi_{i,p}(x), \Phi_{i,q}(x)) \leq k * d_P(p,q)
    \end{equation*}
  \end{itemize}
  \begin{block}{Satz (Barnsley, 1992)}
    Der Attraktor \( A_p \in \mathcal{H}(X) \) ist stetig von \( p \) mittels der Hausdorff-Metrik \( h(d) \) abhängig.
  \end{block}
\end{frame}

\section{Bildkomprimierung}

\begin{frame}{Warum Bildkomprimierung?}
  \begin{itemize}
    \item Auge unempfindlich gegen viele Informationsverluste
    \item Weniger Speicherverbrauch \( \to \) billiger zu speichern
    \item Weniger Speicherverbrauch \( \to \) schneller zu übertragen
  \end{itemize}
\end{frame}

\begin{frame}{Fraktale Bildkomprimierung}
  \textbf{Idee}: Farn kann durch 4 Transformationen erzeugt werden

  \( \to \) es müssen nur 18 Zahlen gespeichert werden \( \Rightarrow \) \alert{\( \sim \) 18 Byte}

  Konventionelle Speicherung: 1 Byte für 8 Pixel \( \Rightarrow \) \alert{500 mal mehr} (10kB)!\footnote{\scriptsize{\textcolor{gray}{200*400 Pixel, je höher desto mehr Bytes werden gebraucht}}}
  
  \  \\

  \pause{}
  \textbf{Problem}:
  \begin{itemize}
    \item IFS \( \to \) Bild: klar
    \item Bild \( \to \) IFS: Wie?
  \end{itemize}
\end{frame}

\begin{frame}{Fraktale Bildkomprimierung --- PIFS}
  Wir rüsten IFS auf --- jede Kontraktion bekommt
  \begin{enumerate}
    \item neue Vorschrift; adaptiert Grauwert
    \item Maske; gibt an, welcher Teil des Urbilds abgebildet wird
  \end{enumerate}
  \pause{}
  \begin{align*}
    \Phi_i: \square \supset D_i &\to w_i(D_i) \eqqcolon R_i \subset \square \\
    \Phi_i(x,y,z) &= \begin{pmatrix}
      a_i & b_i & 0 \\
      c_i & d_i & 0 \\
      0 & 0 & \alert{g_i}
    \end{pmatrix} * \begin{pmatrix}
      x \\ y \\ z
    \end{pmatrix} + \begin{pmatrix}
      e_i \\ f_i \\ \alert{h_i}
    \end{pmatrix}
  \end{align*}
  \textbf{Einschränkungen}:
  \begin{itemize}
    \item \( \bigcup_{i=1}^n R_i = \mathcal{I} \)
    \item \( i \neq j \Rightarrow R_i \cap R_j = \varnothing \)
  \end{itemize}
\end{frame}

\begin{frame}{Fraktale Bildkomprimierung --- PIFS bestimmen}
  \begin{enumerate}
    \item Bild in 8*8-Quadrate unterteilen (\( R \coloneqq \left \{ R_1, \dots, R_n \right \} \), nicht-überlappend)
    \item Bild in 16*16*-Quadrate unterteilen (\( D \coloneqq \left \{ D_1, \dots, D_m \right \} \), überlappend)
    \item Für alle \( R_i \)-Transformationen passendes \( D_j \) suchen und eins wählen
    \item Pro Zuordnung Belichtung und Kontrast anpassen
  \end{enumerate}
\end{frame}

\begin{frame}{Fraktale Bildkomprimierung --- PIFS bestimmen}
  \begin{figure}
    \includegraphics[width=.65\textwidth]{lena-pifs}
    \caption*{\tiny{Originalbild und Lena-PIFS nach der ersten, zweiten und zehnten Iteration}}
  \end{figure}
\end{frame}

\begin{frame}{Fraktale Bildkomprimierung --- Tuning}
  Quadrate-basierter Algorithmus relativ naiv, erzeugt aber passable Ergebnisse

  \textbf{Verbesserungen}:
  \begin{itemize}
    \item mehr als nur 2 Größen von Quadraten
    \item andere geometrische Formen
    \item \dots\  (aktuelle Forschung)
  \end{itemize}
\end{frame}

\begin{frame}[fragile]{Quellen}
  \begin{itemize}
    \item Barnsley --- \textbf{Fractals Everywhere} (1993)
    \item Barnsley, Hurd --- \textbf{Fractal Image Compression} (1993)
    \item Fisher --- \textbf{Fractal Image Compression} (1992)
  \end{itemize}
\end{frame}

\end{document}
